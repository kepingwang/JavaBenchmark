\documentclass[11pt]{article}

\usepackage[bottom]{footmisc}
\usepackage{courier}
\usepackage{graphicx}
\usepackage[hidelinks]{hyperref}
\usepackage{amsmath,amsfonts,amssymb,geometry,bm}
\geometry{top=2cm}
\usepackage{enumerate}
\usepackage{listings}
\usepackage[T1]{fontenc}

\usepackage{color}

\definecolor{codegreen}{rgb}{0,0.6,0}
\definecolor{codegray}{rgb}{0.5,0.5,0.5}
\definecolor{codepurple}{rgb}{0.58,0,0.82}
\definecolor{backcolour}{rgb}{0.95,0.95,0.92}

\lstdefinestyle{mystyle}{
  backgroundcolor=\color{backcolour},
  commentstyle=\color{codegreen},
  keywordstyle=\color{magenta},
  numberstyle=\tiny\color{codegray},
  stringstyle=\color{codepurple},
  basicstyle=\footnotesize,
  breakatwhitespace=false,
  breaklines=true,
  captionpos=b,
  keepspaces=true,
  numbers=left,
  numbersep=5pt,
  showspaces=false,
  showstringspaces=false,
  showtabs=false,
  tabsize=2  
}

\lstset{style=mystyle}

% -------- Title Info
\title{CS201 DNA Project Analysis\footnote{See \href{https://github.com/kepingwang/JavaBenchmark}{https://github.com/kepingwang/JavaBenchmark} for the full benchmarking project.}}
\author{Keping Wang (netid: kw238)}


% -------- The Document
\begin{document}
\maketitle

\section{Benchmarking Method}
I use the JMH framework for benchmarking. Please refer to my \href{https://github.com/kepingwang/JavaBenchmark/blob/master/result/analysis/analysis.pdf}{\underline{Markov Project Analysis}} for detailed explanation.

I generate the testing data with the simplest method. The enzyme is always a single character ``B''. The splicee is a sequence of character ``S'' with length $S$. The source is the concatenation of $b$ ``B''s and $(N-b)$ ``x''s.

For example, with $N=10$, $b=2$, $S=3$, the \texttt{cutAndSplice()} method (for \texttt{StringStrand}) would be:
\begin{lstlisting}[language=Java]
  String source = "BBxxxxxxxx";
  String enzyme = "B";
  String splicee= "SSS";
  (new StringStrand(source)).cutAndSplice(enzyme, splicee);
\end{lstlisting}

\section{Non Linked List Hypotheses}

From the following two figures, we know that the \texttt{cutAndSplice()} method of \texttt{StringStrand} is $O(b^2S)$.

\centerline{\includegraphics[width=0.6\linewidth]{StringStrand_N100000_xb_all.pdf}}

\centerline{\includegraphics[width=0.6\linewidth]{StringStrand_N100000_xS_all.pdf}}

From the following two figures, we know that the \texttt{cutAndSplice()} method of \texttt{StringBuilderStrand} is $O(bS)$.

\centerline{\includegraphics[width=0.6\linewidth]{StringBuilderStrand_N100000_xb_all.pdf}}

\centerline{\includegraphics[width=0.6\linewidth]{StringBuilderStrand_N100000_xS_all.pdf}}

From the following two figures, we confirm that $b$ and $S$ are multiplied in the run time of non linked list strands because the run time is quadratic/linear in $b$ even when $b$ is very small relative to $S$.

\centerline{\includegraphics[width=0.6\linewidth]{StringStrand_N100000_xb.pdf}}

\centerline{\includegraphics[width=0.6\linewidth]{StringBuilderStrand_N100000_xb.pdf}}

Finally the running time of \texttt{cutAndSplice()} for both \texttt{StringStrand} and \texttt{StringBuilderStrand} have to be at least $O(n)$, since the method has to go through the whole strand to find pieces that matches the enzyme.

\section{LinkStrand Hypothesis}

From the following figure, we can see that the run time of \texttt{cutAndSplice()} int \texttt{LinkStrand} is irrelevant to $S$ (of course when $S$ is much smaller than $N$).

\centerline{\includegraphics[width=0.6\linewidth]{LinkStrand_N1000000_xS.pdf}}

From the two following figures, we notice that the run time of \texttt{cutAndSplice()} int \texttt{LinkStrand} is linear in both $N$ and $b$. Besides, when $b$ becomes very small relative to $N$, the run time is dominated by $N$, so we know that the run time is $O(b+N)$ (instead of $O(bN)$).

\centerline{\includegraphics[width=0.6\linewidth]{LinkStrand_S10000_xN_all.pdf}}

\centerline{\includegraphics[width=0.6\linewidth]{LinkStrand_S10000_xN.pdf}}

\centerline{\includegraphics[width=0.6\linewidth]{LinkStrand_S10000_xb_all.pdf}}


\end{document}